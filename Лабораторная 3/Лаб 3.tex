\documentclass[a4paper, 12pt]{article}

\usepackage[utf8]{inputenc}

\usepackage{amsmath, amssymb}

\usepackage[T2A]{fontenc}
\usepackage[english, russian]{babel}

\usepackage[top = 20 mm,
            bottom = 20 mm,
            left = 30 mm,
            right = 30 mm]{geometry}
            
\usepackage{indentfirst}

\setlength{\parindent}{12.5 mm}

\usepackage{setspace}
\setstretch{1}

\setcounter{equation}{61}
 
\begin{document}

\begin{center}
{\bf УРАВНЕНИЯ ГИПЕРБОЛИЧЕСКОГО ТИПА}
\end{center}

Из уравнения  (80) и условий  (81) находим:
$$
X_{1}(x) = C \sin\frac{\omega}{\alpha}x, X_{2}(x) = D \sin\frac{\omega}{\alpha}(l-x);
$$
условия сопряжения  (82) дают:
$$
\begin{aligned}
C \sin\frac{\omega}{\alpha}x_{0} - D \sin\frac{\omega}{\alpha}(l-x_{0}) = 0,\\
C\frac{\omega}{\alpha}\cos\frac{\omega}{\alpha}x_{0} + D \frac{\omega}{\alpha}\cos\frac{\omega}{\alpha}(l-x_{0}) =\frac{A}{\kappa}.
\end{aligned}
$$
Определяя отсюда коэфициенты C и D, получаем:
\begin{equation*}
u(x, t) = 
 \begin{cases}
   u_{1} = \frac{Aa}{\kappa\omega}\frac{\sin\frac{\omega}{\alpha}(l-x_{0})}{\sin\frac{\omega}{\alpha}l}\sin\frac{\omega}{\alpha}x\cos(\omega t) \text{ при } 0\leqslant x \leqslant x_{0}, 
   \\
   u_{2} = \frac{Aa}{\kappa\omega}\frac{\sin\frac{\omega}{\alpha}x_{0}}{\sin\frac{\omega}{\alpha}l}\sin\frac{\omega}{\alpha}(l - x)\cos(\omega t) \text{ при } x_{0}\leqslant x \leqslant 1.
 \end{cases}
\end{equation*}
Аналогично записывается решение при  $f(t) = A\sin\omega t$.

Итак, получено решение для случая $f(t) = A\cos\omega t$ или $f(t) = A\sin\omega t$. Если f(t) - периодическая функция, равная
$$
f(t) = \frac{\alpha_{u}}{2} + \sum\limits_{n=1}^{\infty}(a_{n}\cos(\omega n t) + \beta_{n}\sin(\omega n t)
$$
($\omega$ - наименьшая частота), то, очевидно,
\begin{equation*}
u(x, t) = 
 \begin{cases}
u_{1} = \frac{1}{k}
 \begin{cases}
 \frac{\alpha_{0}x}{2}(1 - \frac{x_{0}}{l}) +  \sum\limits_{n=1}^{\infty}\frac{\alpha\sin\frac{\omega n}{\alpha}(l - x_{0})}{\omega n\sin\frac{\omega n}{\alpha}l}\sin\frac{\omega nx}{\alpha}\times
   \end{cases}
   \\
\times(a_{n}\cos(\omega n t) + \beta_{n}\sin(\omega n t),\text{ } 0\leqslant x \leqslant x_{0}
\\
u_{2} = \frac{1}{k}
 \begin{cases}
   \frac{\alpha_{0}x_{0}}{2}(1 - \frac{x_{0}}{l}) +  \sum\limits_{n=1}^{\infty}\frac{\alpha\sin\frac{\omega n}{\alpha}l}{\omega n\sin\frac{\omega n}{\alpha}l}\sin\frac{\omega n(l - x)}{\alpha}\times
 \end{cases}
\\
\times(a_{n}\cos(\omega n t) + \beta_{n}\sin(\omega n t),\text{ } x_{0}\leqslant x \leqslant 1
 \end{cases}
\end{equation*}

\newpage

\begin{center}
{\bf МЕТОД РАЗДЕЛЕНИЯ ПЕРЕМЕННЫХ}
\end{center}
Если функция f(t) непериодическая, то, представляя её в виде интерала Фурье, аналогичным методом можно получить решение в итегральной форме.

Если знаменатель y этих функций (83) равен нулю
$$
\begin{aligned}
\sin\frac{\omega nl}{\alpha} = 0,\\
\omega n = \frac{\pi m}{l} = \omega_{m}.
\end{aligned}
$$
т.е. если спектр частот возбуждающей силы содержит одну из частот собственных колебаний (резонанс), то установившегося решения не существует.

Если точка приложения силы $x_{0}$ является одним из узлов стоячей волны, соответсвующей свободному колебанию с частотой $\omega_{m}$, то
$$
\begin{aligned}
\sin\frac{\omega_{m}}{\alpha}x_{0} = 0,\\
\sin\frac{\omega_{m}}{\alpha}(l - x_{0}) = 0.
\end{aligned}
$$
При этом числители соответствующих слагаемых для u обращаются в нуль, и явление резонанса не имеет места. Если же точка приложения силы, действующей с частотой $\omega_{m}$, является пучностью соответсвующей стоячей волны с частотой $\omega_{m}$, то
$$
\sin\frac{\omega_{m}{\alpha}}x_{0} = 1,
$$
и явление резонанса будет выражено наиболее резко.

Отсюда следует правило, что для возбуждения резонанса струны при действии на неё сосредоточенной силой надо. чтобы частота её $\omega$ была равна одной из собственных частот струны, а точка приложения силы совпадала с одной из пучностей стоячей волны.

\textbf{9. Общая схема метода разделения переменных.} Метод разделения переменных применим не только для уравнения колебаний однородной струны, но и для уравнения колебаний неоднородной струны. Рассмотрим следующую задачу:

\textit{найти решение уравнения}
$$
$$
\textit{удовлетворяющее условиям}
\end{document}